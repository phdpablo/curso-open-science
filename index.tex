% Options for packages loaded elsewhere
\PassOptionsToPackage{unicode}{hyperref}
\PassOptionsToPackage{hyphens}{url}
\PassOptionsToPackage{dvipsnames,svgnames,x11names}{xcolor}
%
\documentclass[
  a4paper,
]{article}

\usepackage{amsmath,amssymb}
\usepackage{iftex}
\ifPDFTeX
  \usepackage[T1]{fontenc}
  \usepackage[utf8]{inputenc}
  \usepackage{textcomp} % provide euro and other symbols
\else % if luatex or xetex
  \usepackage{unicode-math}
  \defaultfontfeatures{Scale=MatchLowercase}
  \defaultfontfeatures[\rmfamily]{Ligatures=TeX,Scale=1}
\fi
\usepackage{lmodern}
\ifPDFTeX\else  
    % xetex/luatex font selection
\fi
% Use upquote if available, for straight quotes in verbatim environments
\IfFileExists{upquote.sty}{\usepackage{upquote}}{}
\IfFileExists{microtype.sty}{% use microtype if available
  \usepackage[]{microtype}
  \UseMicrotypeSet[protrusion]{basicmath} % disable protrusion for tt fonts
}{}
\makeatletter
\@ifundefined{KOMAClassName}{% if non-KOMA class
  \IfFileExists{parskip.sty}{%
    \usepackage{parskip}
  }{% else
    \setlength{\parindent}{0pt}
    \setlength{\parskip}{6pt plus 2pt minus 1pt}}
}{% if KOMA class
  \KOMAoptions{parskip=half}}
\makeatother
\usepackage{xcolor}
\usepackage[paperwidth=8.00in,paperheight=10.00in,left=1.25in,textwidth=
5.25in,top=1.00in,textheight=8.25in]{geometry}
\setlength{\emergencystretch}{3em} % prevent overfull lines
\setcounter{secnumdepth}{5}
% Make \paragraph and \subparagraph free-standing
\ifx\paragraph\undefined\else
  \let\oldparagraph\paragraph
  \renewcommand{\paragraph}[1]{\oldparagraph{#1}\mbox{}}
\fi
\ifx\subparagraph\undefined\else
  \let\oldsubparagraph\subparagraph
  \renewcommand{\subparagraph}[1]{\oldsubparagraph{#1}\mbox{}}
\fi


\providecommand{\tightlist}{%
  \setlength{\itemsep}{0pt}\setlength{\parskip}{0pt}}\usepackage{longtable,booktabs,array}
\usepackage{calc} % for calculating minipage widths
% Correct order of tables after \paragraph or \subparagraph
\usepackage{etoolbox}
\makeatletter
\patchcmd\longtable{\par}{\if@noskipsec\mbox{}\fi\par}{}{}
\makeatother
% Allow footnotes in longtable head/foot
\IfFileExists{footnotehyper.sty}{\usepackage{footnotehyper}}{\usepackage{footnote}}
\makesavenoteenv{longtable}
\usepackage{graphicx}
\makeatletter
\def\maxwidth{\ifdim\Gin@nat@width>\linewidth\linewidth\else\Gin@nat@width\fi}
\def\maxheight{\ifdim\Gin@nat@height>\textheight\textheight\else\Gin@nat@height\fi}
\makeatother
% Scale images if necessary, so that they will not overflow the page
% margins by default, and it is still possible to overwrite the defaults
% using explicit options in \includegraphics[width, height, ...]{}
\setkeys{Gin}{width=\maxwidth,height=\maxheight,keepaspectratio}
% Set default figure placement to htbp
\makeatletter
\def\fps@figure{htbp}
\makeatother
% definitions for citeproc citations
\NewDocumentCommand\citeproctext{}{}
\NewDocumentCommand\citeproc{mm}{%
  \begingroup\def\citeproctext{#2}\cite{#1}\endgroup}
\makeatletter
 % allow citations to break across lines
 \let\@cite@ofmt\@firstofone
 % avoid brackets around text for \cite:
 \def\@biblabel#1{}
 \def\@cite#1#2{{#1\if@tempswa , #2\fi}}
\makeatother
\newlength{\cslhangindent}
\setlength{\cslhangindent}{1.5em}
\newlength{\csllabelwidth}
\setlength{\csllabelwidth}{3em}
\newenvironment{CSLReferences}[2] % #1 hanging-indent, #2 entry-spacing
 {\begin{list}{}{%
  \setlength{\itemindent}{0pt}
  \setlength{\leftmargin}{0pt}
  \setlength{\parsep}{0pt}
  % turn on hanging indent if param 1 is 1
  \ifodd #1
   \setlength{\leftmargin}{\cslhangindent}
   \setlength{\itemindent}{-1\cslhangindent}
  \fi
  % set entry spacing
  \setlength{\itemsep}{#2\baselineskip}}}
 {\end{list}}
\usepackage{calc}
\newcommand{\CSLBlock}[1]{\hfill\break\parbox[t]{\linewidth}{\strut\ignorespaces#1\strut}}
\newcommand{\CSLLeftMargin}[1]{\parbox[t]{\csllabelwidth}{\strut#1\strut}}
\newcommand{\CSLRightInline}[1]{\parbox[t]{\linewidth - \csllabelwidth}{\strut#1\strut}}
\newcommand{\CSLIndent}[1]{\hspace{\cslhangindent}#1}

\makeatletter
\@ifpackageloaded{bookmark}{}{\usepackage{bookmark}}
\makeatother
\makeatletter
\@ifpackageloaded{caption}{}{\usepackage{caption}}
\AtBeginDocument{%
\ifdefined\contentsname
  \renewcommand*\contentsname{Table of contents}
\else
  \newcommand\contentsname{Table of contents}
\fi
\ifdefined\listfigurename
  \renewcommand*\listfigurename{List of Figures}
\else
  \newcommand\listfigurename{List of Figures}
\fi
\ifdefined\listtablename
  \renewcommand*\listtablename{List of Tables}
\else
  \newcommand\listtablename{List of Tables}
\fi
\ifdefined\figurename
  \renewcommand*\figurename{Figure}
\else
  \newcommand\figurename{Figure}
\fi
\ifdefined\tablename
  \renewcommand*\tablename{Table}
\else
  \newcommand\tablename{Table}
\fi
}
\@ifpackageloaded{float}{}{\usepackage{float}}
\floatstyle{ruled}
\@ifundefined{c@chapter}{\newfloat{codelisting}{h}{lop}}{\newfloat{codelisting}{h}{lop}[chapter]}
\floatname{codelisting}{Listing}
\newcommand*\listoflistings{\listof{codelisting}{List of Listings}}
\makeatother
\makeatletter
\makeatother
\makeatletter
\@ifpackageloaded{caption}{}{\usepackage{caption}}
\@ifpackageloaded{subcaption}{}{\usepackage{subcaption}}
\makeatother
\ifLuaTeX
  \usepackage{selnolig}  % disable illegal ligatures
\fi
\usepackage{bookmark}

\IfFileExists{xurl.sty}{\usepackage{xurl}}{} % add URL line breaks if available
\urlstyle{same} % disable monospaced font for URLs
\hypersetup{
  pdftitle={Artigo à Prova de Futuro},
  pdfauthor={Pablo Rogers},
  colorlinks=true,
  linkcolor={blue},
  filecolor={Maroon},
  citecolor={Blue},
  urlcolor={Blue},
  pdfcreator={LaTeX via pandoc}}

\title{Artigo à Prova de Futuro}
\usepackage{etoolbox}
\makeatletter
\providecommand{\subtitle}[1]{% add subtitle to \maketitle
  \apptocmd{\@title}{\par {\large #1 \par}}{}{}
}
\makeatother
\subtitle{Jornada de Open Science na Prática}
\author{Pablo Rogers}
\date{March 1, 2024}

\begin{document}
\maketitle

\bookmarksetup{startatroot}

\section*{HOME}\label{home}

\markboth{HOME}{HOME}

\subsection*{Sobre essa página}\label{sobre-essa-puxe1gina}

\markright{Sobre essa página}

Página do curso \textbf{``Artigo à Prova de Futuro: Jornada de Open
Science na Prática''}. Aqui você encontrará informações sobre o programa
do curso, materiais para seu acompanhamento e sugestões de leituras
sobre a prática da ciência aberta (artigos, notas de aulas, blogs,
vídeos, etc.).

\subsection*{Sobre o instrutor}\label{sobre-o-instrutor}

\markright{Sobre o instrutor}

O curso será ministrado por Pablo Rogers, doutor em administração pela
Universidade de São Paulo (FEA/USP) e professor de métodos quantitativos
desde 2007. Na \href{https://github.com/phdpablo}{página de perfil do
Github} do instrutor você pode encontrar informações de trabalhos
recentes, e no seu \href{https://phdpablo.com/}{site pessoal}, detalhes
sobre suas formações, competências, trajetória e projetos.

\subsection*{Sobre o curso}\label{sobre-o-curso}

\markright{Sobre o curso}

O curso tem objetivo de introduzir os conceitos relacionados com a
ciência aberta e a prática da pesquisa reprodutível. O curso aborda
temas introdutórios sobre ciência aberta, com foco no ferramental
disponível para tornar a pesquisa mais transparente, reprodutível e
acessível. O curso é voltado para pesquisadores e estudantes de
pós-graduação, mas aberto a qualquer pessoa interessada em aprender
sobre a prática da ciência aberta. O protagonista do curso é o
pesquisador brasileiro que deseja aprimorar a qualidade e a
transparência de sua pesquisa, e que busca ferramentas para tornar-lá
mais eficiente e acessível.

Trata-se de um curso intermitente programado para acontecer em 4
encontros de 4 horas cada (ou 8 encontros de 2 horas cada), totalizando
16 horas de aulas síncronas. Ele acontecerá algumas vezes ao ano, com
datas e horários a serem definidos.

O curso é gratuito, com a possibilidade de certificado de extensão pela
Universidade Federal de Uberlândia (UFU), e as inscrições serão feitas
por meio de formulário disponibilizado no site do projeto
Psico\&Econo\_METRIA. Quando da previsão das datas, uma campanha de
e-mail marketing divulgará o link para a inscrição através de
coordenações de pós-graduações selecionadas.

As vagas são limitadas e a seleção será feita por ordem de inscrição.
Após o preenchimento das vagas, os demais interessados poderão se
inscrever em uma lista de espera, para serem avisados sobre a próxima
edição do curso. Após selecionados, os inscritos receberão um e-mail com
instruções para acesso à plataforma de aulas síncronas e para a
realização das atividades prévias ao curso.

\subsection*{Ementa do curso}\label{ementa-do-curso}

\markright{Ementa do curso}

Introdução da Ciência Aberta / Repositórios da Ciência Aberta /
Gerenciamento de Referências e Bibliotecas / Gestão de Dados e Projetos
/ Controle de Versão / Documentos Reprodutíveis / Controle de Ambiente
(containers) / IA Aplicada à Pesquisa Científica

\subsection*{Metodologia}\label{metodologia}

\markright{Metodologia}

O curso sempre acontecerá de forma remota e síncrona, com aulas
expositivas e práticas. As aulas serão gravadas e disponibilizadas no
\href{https://www.youtube.com/c/PsicoEconoMETRIA}{canal do YouTube do
projeto Psico\&Econo\_METRIA}. Nesse sentido, o material do curso
organizado nessa página refere-se ao roteiro estruturado de tudo que se
vê nas aulas síncronas e conteúdos adicionais (bibliografia, notas de
aulas, links, etc).

\bookmarksetup{startatroot}

\section*{Pré-requisitos}\label{pruxe9-requisitos}

\markboth{Pré-requisitos}{Pré-requisitos}

O curso não exige conhecimento prévio em programação, mas é recomendável
que o aluno tenha familiaridade com o uso de computadores (ambiente
Windows) e com a escrita de textos científicos. Nesse sentido, não é
necessário ter conhecimento prévio sobre as ferramentas e plataformas
que utilizaremos no curso: Zotero, OSF, Zenodo, Git, Github, RStudio,
Quarto/RMarkdown, Docker, etc; mas desejável que o aluno já as tenha
instalado e/ou cadastro nas plataformas.

Abaixo eu descrevo sucintamente o que é cada uma dessas ferramentas e
plataformas, e como você pode se preparar para o curso. Também apresento
um vídeo curto sobre a instalação e cadastro em cada uma delas. A ideia
é que você já tenha todas as ferramentas e plataformas instaladas e/ou
cadastro antes do início do curso, para que possamos focar no conteúdo e
prática durante as aulas síncronas. Mas pode ficar tranquilo, pois na
primeira aula do curso abordaremos essas tarefas, e caso ainda haja
alguma dúvida na instalação e cadastro, dedicaremos algum tempo para
saná-las.

Outras soluções que iremos discutir e testar durante o curso, como
alguns pacotes do R, e aplicações de IA no último módulo, deixaremos
para as aulas remotas. Essas soluções na sua maioria requerem cadastros
rápidos, e podem ser feitos de forma instantânea via conta
Google/Microsoft/Apple.

\subsection*{Github}\label{github}
\addcontentsline{toc}{subsection}{Github}

\markright{Github}

Primeiramente, se cadastre no Github: \url{https://github.com/signup},
pois com ele você poderá acessar o material do curso e interagir com os
demais participantes. E com a conta do Github você também poderá se
cadastrar em outras plataformas, como o Zenodo, OSF, etc. Algumas
features que aprenderemos no curso exigem o vínculo entre as contas. Se
for professor ou estudante, você pode solicitar o
\href{https://education.github.com/}{GitHub Education} e ter acesso, por
exemplo, ao Copilot, uma das ferramentas de IA que abordaremos no último
módulo. Por isso, é importante que você se cadastre com um e-mail
institucional. Use o mesmo e-mail para se cadastrar em todas
plataformas.

\subsection*{Git}\label{git}
\addcontentsline{toc}{subsection}{Git}

\markright{Git}

Github não é a mesma coisa que Git. O Github é uma plataforma, e o Git é
uma ferramenta. Instale a versão mais recente do Git:
\url{https://git-scm.com/downloads}. O Git é uma ferramenta de controle
de versão, e o Github é uma plataforma que utiliza o Git. O Git é uma
ferramenta essencial para a prática da ciência aberta, e é uma das
ferramentas mais importantes para o pesquisador que deseja tornar sua
pesquisa mais transparente e reprodutível.

\subsection*{Zotero}\label{zotero}
\addcontentsline{toc}{subsection}{Zotero}

\markright{Zotero}

Baixe a versão mais recente do Zotero:
\url{https://www.zotero.org/download/} e cadastre uma conta:
\url{https://www.zotero.org/user/register/}. Vamos discutir sobre o
Zotero e diversos plugins que são úteis no dia-a-dia do pesquisador.
Atualmente, o Zotero é a ferramenta mais completa para gerenciamento de
referências e bibliotecas, e se integra nativamente com o RStudio.

\subsection*{OSF}\label{osf}
\addcontentsline{toc}{subsection}{OSF}

\markright{OSF}

Cadastre no Open Science Framework (OSF):
\url{https://osf.io/register/}. Como veremos, essa plataforma é uma das
mais importantes para a prática da ciência aberta. Ela está no começo
(pré-registro) e no final (repositório de dados e pré-print) do ciclo de
vida (workflow) de um projeto de pesquisa.

\subsection*{Zenodo}\label{zenodo}
\addcontentsline{toc}{subsection}{Zenodo}

\markright{Zenodo}

Apesar do Zenodo cumprir funções similares ao OSF e até mesmo ao Github,
ele é mais voltado para a publicação de dados e publicações científicas.
Cadastre no Zenodo: \url{https://zenodo.org/login/} e víncule sua conta
com o Github. Isso será útil, principalmente, para geração de DOI de
repositórios do Github.

\subsection*{RStudio}\label{rstudio}
\addcontentsline{toc}{subsection}{RStudio}

\markright{RStudio}

Baixe a versão mais recente do RStudio:
\url{https://posit.co/download/rstudio-desktop/}. O RStudio é uma
Integrated Development Environment (IDE) para a linguagem R. O RStudio é
uma ferramenta essencial para a prática da ciência aberta em R, pois
integra as principais soluções que abordaremos no curso (Zotero, Quarto,
Git/Github, etc.). A empresa RStudio recentemente mudou o nome para
Posit, com o objetivo refletir melhor a expansão da empresa para além do
desenvolvimento de ferramentas para R, incluindo Python e outras
linguagens. Nesse mesmo link você pode baixar o R, que é a linguagem de
programação que utilizaremos no curso.

\subsection*{Quarto}\label{quarto}
\addcontentsline{toc}{subsection}{Quarto}

\markright{Quarto}

Baixe a versão mais recente do Quarto: \url{https://www.quarto.org/}. O
Quarto é uma linguagem de marcação que permite a criação de documentos
reprodutíveis e dinâmicos. Ele é uma evolução e tende a substituir o
RMarkdown, que é a principal linguagem de marcação do R. O Quarto
engloba e adiciona diversas outras vantagens ao RMarkdown, tal como a
possibilidade de criar documentos reprodutíveis em Python, Julia, etc.
Se você já tem algum conhecimento de RMarkdown, não se preocupe, pois o
Quarto é uma extensão natural.

\subsection*{Docker}\label{docker}
\addcontentsline{toc}{subsection}{Docker}

\markright{Docker}

Baixe a versão mais recente do Docker:
\url{https://www.docker.com/products/docker-desktop}. Nesse mesmo link
você cria uma conta. O Docker é uma plataforma para desenvolvimento,
envio e execução de aplicativos. O Docker é uma ferramenta essencial
para a prática da ciência aberta, pois permite a criação de ambientes
reprodutíveis.

\bookmarksetup{startatroot}

\section{Introdução à Ciência
Aberta}\label{introduuxe7uxe3o-uxe0-ciuxeancia-aberta}

\bookmarksetup{startatroot}

\section{Repositórios da Ciência
Aberta}\label{reposituxf3rios-da-ciuxeancia-aberta}

\bookmarksetup{startatroot}

\section{Gerenciamento de Referências e
Bibliotecas}\label{gerenciamento-de-referuxeancias-e-bibliotecas}

\bookmarksetup{startatroot}

\section{Gestão de Dados e
Projetos}\label{gestuxe3o-de-dados-e-projetos}

\bookmarksetup{startatroot}

\section{Controle de versão}\label{controle-de-versuxe3o}

\bookmarksetup{startatroot}

\section{Documentos Reprodutíveis}\label{documentos-reprodutuxedveis}

\bookmarksetup{startatroot}

\section{Controle de Ambiente
(Containers)}\label{controle-de-ambiente-containers}

\bookmarksetup{startatroot}

\section{IA Aplicada à Pesquisa
Científica}\label{ia-aplicada-uxe0-pesquisa-cientuxedfica}

\bookmarksetup{startatroot}

\section*{Referências}\label{referuxeancias}
\addcontentsline{toc}{section}{Referências}

\markboth{Referências}{Referências}

\phantomsection\label{refs}
\begin{CSLReferences}{0}{1}
\end{CSLReferences}



\end{document}
